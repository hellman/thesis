\section{Introduction}

\emph{Cryptography} is the science of secure communication and storage. What does ``secure'' mean in this definition? First, it means that, apart from the two communicating parties, there may be a third party trying to learn confidential information, to disrupt the communication, or to mislead the communicating parties. Second, ``secure'' means that a predefined set of goals cannot be achieved by any adversary with set capabilities, such as an ability to read or modify the communications, or limitations, such as having limited computational power.

Cryptography is often divided into design and cryptanalysis. \emph{Cryptographic design} is the design of secure communication systems, called \emph{cryptosystems}. \emph{Cryptanalysis} is \emph{breaking} the security of cryptosystems. In order to understand the security of a cryptosystem better, simplified versions of the cryptosystem are often analyzed. Of course, cryptography and cryptanalysis are not independent. Design of a cryptosystem usually follows alternating steps: design - cryptanalysis - design - cryptanalysis - ..., until the designers can not cryptanalyze the cryptosystem. After that, the cryptosystem is published and for others to analyze. 

Modern cryptography is broadly split into private-key and public-key cryptography, also called symmetric-key and asymmetric-key cryptography.

\emph{Symmetric-key cryptography} assumes that the communicating parties have a shared private key. For example, they could meet physically and agree on the common secret key. In this case, the same shared key can be used both for encrypting and decrypting communications. 

Symmetric-key cryptosystems are usually constructed from low-level, bitwise operations and functions with small domains. They are very efficient. However, their security is not mathematically proven and is not based on any simple mathematical problem.

\emph{Asymmetric-key cryptography} does not necessarily require pre-shared keys. The defining property, however, is that the key may consist of two parts - a public key and a private key. For example, the public key may be used for encrypting messages and may be openly published. The private key is then used for decrypting the messages and must be kept secret.
Asymmetric-key cryptography can also be used to establish a shared secret securely while using an insecure channel.

Asymmetric-key cryptosystems are usually constructed around mathematical structures from number theory or algebraic geometry. Most often these cryptosystems are relatively inefficient. However, their security is based on the hardness of solving a mathematical problem, such as factoring large integers. It means that, if the cryptosystem is cryptanalyzed and broken, then it would mean that the underlying mathematical problem is not hard and can be solved efficiently.

The Public-key cryptography is a very rich area which gives rise to many beautiful cryptosystems and protocols. It is a very active field and many challenging problems are continuously solved and new ones are identified. In this dissertation, however, I dive into symmetric-key cryptography and do not study public-key cryptosystems. As an exception, in \PartRef{wb}, I study white-box cryptography, which, in particular, aims to construct a public-key cryptosystem from a symmetric-key primitive. 

In practice, a hybrid method is used. The public-key cryptography is used to establish a shared secret key between the communicating parties, and all the consequent communications are encrypted using fast symmetric-key cryptography.


\subsection{Authenticated Encryption}

The main goal of symmetric cryptography is to provide \emph{authenticated encryption}. Authenticated encryption is a cryptosystem providing \emph{confidentiality}, \emph{integrity}, and \emph{authenticity}.
\begin{itemize}
    \item \emph{Confidentiality} guarantees that any adversary with predefined capabilities cannot recover any information about the original messages (called plaintexts) from the encrypted messages (called ciphertexts).
    
    \item \emph{Integrity} guarantees that any adversary with predefined capabilities cannot modify a transmitted ciphertext without the change being noticed by the receiver.
    
    \item \emph{Authenticity} guarantees that the receiving party can be assured that the message was generated by the sender.
\end{itemize}

There are several ways to construct an authenticated encryption scheme.

\subsubsection{Authenticated Encryption from Block Ciphers}

Block ciphers are the classical and the most widely used symmetric-key primitives. Formally, a block cipher is a family of permutations, where the secret key selects one of the permutations. The domain of the permutation is the message space. 

The most widely spread block cipher is the AES block cipher~\cite{AES}, also called Rijndael, designed by Vincent Rijmen and Joan Daemen. It was standardized in 2001 by the US standardization agency NIST.

I and my colleagues designed a block cipher called SPARX~\cite{OurSPARX}. The design process and analysis are described in \ChapRef{sparx}.

A plain block cipher can only encrypt fixed-width messages. For example, AES has a 128-bit block size. The bigger problem is that direct encryption of message blocks under the same key (i.e., by the same permutation) leaks information about the equality of message blocks: if the two plaintext blocks are equal, then the two ciphertext blocks are equal too, which contradicts the confidentiality requirement.
Another big problem is that authenticity is not guaranteed. The blocks can be removed arbitrarily without being noticed.

An \emph{authenticated block cipher mode} is a construction that uses a block cipher to create an authenticated encryption scheme. Such a construction often consists of two parts: an encryption scheme for confidentiality and a message authentication code (MAC) for integrity. For example, the Encrypt-then-MAC~\cite{EncryptMAC} is a very generic mode that can combine an arbitrary (secure) encryption scheme and an arbitrary (secure) message authentication code in order to create the authenticated encryption. More specific authenticated encryption modes (e.g. GCM~\cite{GCM}, OCB~\cite{OCB}) partially reuse the computations of the two components and achieve better performance. Another class of modes (e.g. SCT~\cite{SCT}, COPA~\cite{COPA}, POET~\cite{POET}) requires a \emph{tweakable block cipher}~\cite{TBC}. Tweakable block ciphers take as input an extra \emph{public} parameter called a \emph{tweak}, and different tweaks should produce indistinguishable block ciphers.


\subsubsection{Authenticated Encryption from Stream Ciphers}

The \emph{one-time pad} is one of the first modern encryption schemes. It is a very simple cipher, but it is famous for achieving \emph{perfect secrecy}. It means that \emph{no information} is revealed from a ciphertext, even for a computationally-unbounded adversary. This property is also called \emph{information-theoretic} security. The one-time pad accepts an $n$-bit plaintext and an $n$-bit key. It combines the key and the plaintext using the exclusive-or operation. The plaintext and the key bits at the same position are added modulo 2. The requirement, however, is that the key should be sampled uniformly at random and used to encrypt only one message. These restrictions are not very practical and are the price for \emph{perfect secrecy}. Indeed, Shannon~\cite{Shannon} proved that these restrictions are \emph{necessary} if perfect secrecy must be kept.

\emph{Stream ciphers}, similarly to block ciphers, discard the perfect secrecy requirement and aim at more practical cryptosystems. Unlike block ciphers, stream ciphers attempt to simulate the one-time pad by generating the required large key (a \emph{keystream}) from a small key on the fly. The keystream is then called \emph{pseudorandom}. The security is based on the requirement that any (computationally-bounded) adversary cannot distinguish the pseudorandom keystream from a purely random sequence.

As in block ciphers, stream ciphers can be combined with a message authentication code (MAC) to create an authenticated encryption (e.g. ChaCha20-Poly1305~\cite{chachapoly}). Authenticated modes for stream ciphers were studied in~\cite{streamae}. Another approach is to design an authenticated stream cipher from scratch (e.g. ACORN~\cite{ACORN}).


\subsubsection{Authenticated Encryption from Permutations via Sponge construction}

A (cryptographic) \emph{hash function} is a cryptographic primitive that maps a bit-string of arbitrary length into a fixed-length bit-string. For a secure hash function, it should not be computationally easy to invert it (\emph{preimage resistance}) or find two messages that have the same hash value (\emph{collision resistance}). Furthermore, given a fixed message it should be computationally difficult to find another distinct message that has the same hash value (\emph{second preimage resistance}). In general, hash functions are often modeled as \emph{random oracles}. These oracles always return a truly random element of the hash function's codomain, except that for repeated queries with the same message they always return the same hash value. Hash functions are used in a huge number of protocols and public-key constructions. Since hash functions are \emph{keyless}, it is not clear whether they belong to symmetric-key or asymmetric-key cryptography. In practice, hash functions used are made in the symmetric-key style: created from low-level binary operations, very efficient but with heuristic security. However, there exist hash functions from more algebraic constructions, but they are typically only used in theoretic studies due to their inefficiency.

The \emph{sponge} construction was first formally presented in~\cite{sponge-ecrypt}, though similar ideas had already been used before (e.g.~\cite{LEX,RadioGatun}). It was used to design the winner Keccak~\cite{sponge-keccak} of the SHA-3~\cite{sha3} hash function competition organized by NIST. The sponge construction uses a primitive called \emph{cryptographic permutation}. The state is divided into the \emph{rate} part and the \emph{capacity} part. The rate part is usually controlled by an adversary, while the capacity part is uncontrolled. The sponge \emph{absorbs} message blocks in-between calls to the permutation. Afterward, it \emph{squeezes} pseudorandom outputs (e.g., hash values) in-between calls to the permutation. The construction is illustrated in~\FigRef{sponge}. The sponge is a \emph{provably secure} construction: if the chosen cryptographic permutation is \emph{ideal} (e.g., a purely random permutation), then the construction is guaranteed to be secure up to some level.

\FigTex{sponge.tex}

The designers of Keccak further showed~\cite{sponge-duplex,sponge-monkey} that sponges can be used to construct authenticated encryption. The latter variant of the mode is called MonkeyDuplex. This mode and its variants were used in several encryption schemes (e.g.~\cite{NORX,Ketje,ASCON}). Since a sponge only requires a cryptographic permutation, it inspired cryptography designs called \emph{permutation-based cryptography}. A recent improvement is the Beetle mode~\cite{beetle}, which achieves better security bounds.

I and my colleagues designed a hash function family \texttt{Esch} and an authenticated encryption family \texttt{Schwaemm}. They are based on the recent sponge-based mode called Beetle and a cryptographic permutation family derived from our SPARX block cipher. These designs are submitted to the NIST Call for Lightweight Cryptography~\cite{NISTlight}. I describe the design process and analysis of these primitives in \ChapRef{sparkle}.


\subsubsection{The CAESAR Competition}
Recently, the CAESAR competition was organized~\cite{CAESAR}. Its name stands for ``Competition for Authenticated Encryption: Security, Applicability, and Robustness''. The competition started in 2014 when 53 authenticated encryption schemes were submitted. After 5 years of selection process consisting of 3 rounds, the committee selected 8 portfolio members, from which 4 are the preferred choice. The portfolio is split into 3 use cases:

\begin{enumerate}
    \item \emph{Lightweight applications (resource constrained environments).}
    The preferred choice is ASCON~\cite{ASCON} which is based on MonkeyDuplex sponge mode. The second choice is ACORN~\cite{ACORN}, a dedicated authenticated stream cipher.
    
    \item \emph{High-performance applications.}
    The following two choices are chosen without a preference. The first one is AEGIS-128~\cite{AEGIS}, a dedicated design using a reduced-round AES as a component. The second one is OCB~\cite{OCBcaesar}, a block cipher mode.
    
    \item \emph{Defense in depth.}
    The preferred choice is Deoxys-II~\cite{DEOXYS}, an authenticated encryption scheme based on a tweakable block cipher. The alternative choices are COLM~\cite{COLM}, AES-COPA~\cite{COPA}, and ELmD~\cite{ELMD}, block cipher modes.
\end{enumerate}


\subsection{Black, Gray and White-box Models}

Security of cryptosystems is most often analyzed in a \emph{game-based} setting. The game usually happens between a \emph{challenger} and an \emph{adversary}. The challenger possesses secret information, for example, a secret key. The adversary is allowed to ask specified queries to the challenger. The goal of the adversary is to recover the secret information or, at least, a part of it.

Consider an encryption scheme. The challenger flips a coin and decides whether he will use the encryption scheme or its \emph{ideal} equivalent. In the first case, the challenger chooses the secret key uniformly at random. In the second case, the encryption is performed in the best possible way while maintaining the interface and semantics of the encryption scheme. For example, for each plaintext, the ciphertext may be assigned uniformly at random. Note that the challenger is not necessarily an algorithm and usually is not computationally-bounded, unlike the adversary. The game continues. The adversary can ask the challenger to encrypt several plaintexts chosen by the adversary. The challenger performs the encryption (either using the encryption scheme or its idealized version) and gives ciphertexts to the adversary. It is said the adversary is given access to the \emph{encryption oracle}.
The adversary finally has to guess, what the outcome of the challenger's coin flip was. That is, the adversary has to decide, whether the encryption was done using the encryption scheme or using its idealized version. If the adversary succeeds with non-negligible probability, then it is said that the encryption scheme has an  \emph{adaptive chosen-plaintext distinguisher}. If the adversary accesses the encryption oracle only once, it is said that the scheme has a \emph{(non-adaptive) chosen-plaintext distinguisher}.

There are three major models in which cryptosystems are analyzed. 

\subsubsection{The Black-box Model}
The \emph{black-box} model restricts the analysis to the ``functional'' side of cryptosystems. An adversary in this model is usually given access to encryption and/or decryption oracles. That is, the adversary is allowed to ask the challenger to encrypt and/or decrypt arbitrary messages. Any intermediate computations or events are not visible to the adversary, thus the name ``black-box''. This model is fundamental - any weakness in this model is inherited to the gray-box and white-box models.

\subsubsection{The Gray-box Model}
The \emph{gray-box} model studies the ``physical'' side of cryptosystems, more precisely, of their \emph{implementations} and the devices on which the implementation is deployed. Indeed, this side provides much more information to the adversary. The adversary may be allowed to measure the time of execution of a query, the power consumption of the device, the electromagnetic radiation. This information is usually referred to as \emph{side-channel} information.
The adversary may be \emph{active} - for example, introduce faults in the computations, by heating the device or tweaking the voltage. It is an interesting phenomenon that physical access to the device often enables much more efficient attacks on the cryptosystem. Cryptanalysis in this model is called \emph{side-channel} cryptanalysis.

Countermeasures against \emph{side-channel} attacks may be introduced both in the implementation code and on the physical side. Protections that can be added to the implementation are more generic and, therefore, more preferable. In practice, both methods are used to ensure maximum security. Importantly, implementations typically may use (pseudo)randomness in order to protect computations. Together with the noise and uncertainty of the physical observations, these properties allow the creation of sound countermeasures against side-channel attacks.

\subsubsection{The White-box Model}
The \emph{white-box} model considers the extreme case when the adversary has \emph{full} access to the implementation, in the form of compiled code or Boolean circuits. Typically, the implementation contains a secret key and the adversary's main goal is to recover it. The hardness of the key recovery is often called the \emph{weak white-box} requirement. Other goals may be considered, such as compressing the implementation, inverting the computed function, or removing hidden ``watermarks'' allowing the user possessing the implementation to be traced. The respective security properties are called \emph{unbreakability}, \emph{incompressibility}, \emph{one-wayness}, \emph{(traitor) traceability} (see~\cite{wbNotionsOld,wbNotions}).
Unbreakability together with one-wayness result in a public-key scheme, if the embedded secret key allows efficient decryption. Such implementation is also called a \emph{strong white-box}. Indeed, the implementation secure in the white-box model can be seen as a public key, and the embedded secret key can be seen as a private key. A white-box implementation of a common symmetric encryption scheme would then have a very efficient decryption code. However, it turns out to be a challenging, if not impossible problem.

The white-box model was first introduced by Chow~\etal{}~\cite{ChowAES,ChowDES} in 2002. The authors proposed rather efficient white-box implementations of the AES and DES block ciphers. Unfortunately, they were broken with practical attacks. All consequent attempts to fix the scheme failed as well. A secure white-box implementation of a block cipher remains an open problem today.

White-box implementations are closely related to the notion of \emph{cryptographic obfuscation}. Indeed, a basic implementation has to be obfuscated in order to hide the secret key. There is an active research direction related to \emph{indistinguishability obfuscation} (iO), which is widely believed to be ``the best possible'' obfuscation. iO has many applications in theory: it is known that many provably secure cryptographic primitives can be created from secure iO. Unfortunately, many recent iO candidates were broken. Furthermore, all constructions are very inefficient. For example, a recent framework 5Gen-c~\cite{5GEN} can be used to obfuscate only a single round of the AES block cipher. I remark though, that there is no established provable link between white-box and iO.

\subsubsection{The WhibOx Competition}

In 2017, the WhibOx competition~\cite{whibox} was organized. Any person or team in the world could submit a white-box AES-128 implementation in C code of size up to 50 megabytes, then the implementation was publicly available for analysis. The goal was to recover the secret key from the implementation.

Among 94 submissions, most implementations were broken in less than a day. Only 13 implementations required at least one day to be broken, and only 8 of them required at least two days. The winning implementation survived 28 days, and the following implementation only 12 days. The winning design was created by myself and Alex Biryukov. The implementation did not involve any new provable security techniques, but relied on many interesting obfuscation tricks, effectively slowing down the reverse-engineering effort. We were also first to successfully cryptanalyze the best 3 implementations besides ours. Our participation in the competition initiated the research that resulted in~\PartRef{wb} of this thesis.


\subsection{Cryptanalysis of Symmetric-key Primitives}

The framework for cryptanalysis is most developed for block ciphers. Indeed, block ciphers were used from the 1970s with the designs of the LUCIFER and DES block ciphers. Together with a proper mode, a block cipher can be used to construct an authenticated encryption scheme. Furthermore, block ciphers tend to have a reasonably simple structure. This simplicity attracts cryptanalysts, who try to break the cipher using both established and novel methods of cryptanalysis. Since the same low-level operations are used in most symmetric-key primitives, cryptanalysis methods for block ciphers are usually very generic and can be applied to other primitives, such as stream ciphers, hash functions, message authentication codes, and authenticated encryption. 

What does it mean to break a cipher? In the scientific community, successful cryptanalysis means an algorithm that disproves a security claim of the designers. A typical security claim is that the secret key can not be recovered faster than exhaustive search over the whole key space. A block cipher with a fixed secret key should not be distinguishable from a truly random permutation. Even if the attack is impossible in practice, it only matters that it is faster than the generic attack. The reason is that such an attack shows a \emph{weakness} of the block cipher. Since block ciphers are not provably secure, any weakness should be avoided.

The \emph{complexity} of a cryptanalytic attack is measured by the time, memory and data complexities of the algorithm. The data complexity corresponds to the number of queries that it makes. 

In the simplest form, cryptanalytic attacks lead to a \emph{distinguisher} from a random permutation. In most cases, such an attack can be extended into the secret key recovery. This is done by guessing a part of the secret key and decrypting a part of the ciphertext. Then the correctness of the key guess is verified by using the established distinguisher.

\subsubsection{Cryptanalysis Methods}

In \emph{differential} cryptanalysis, an adversary encrypts two plaintexts with a fixed \txor difference. By an analysis of the cipher's structure, the adversary predicts a difference between ciphertexts with high enough probability. More precisely, the cryptanalyst studies the evolution of the plaintexts difference through all computations, until the ciphertext difference. A transition through nonlinear components is usually probabilistic, and all transitions' probabilities accumulate in an approximation of the probability of observing a particular ciphertext difference. 

In \emph{linear} cryptanalysis, an adversary receives many plaintext-ciphertext pairs generated by the analyzed block cipher. The cipher is approximated by \emph{linear} equations, i.e. equations involving only the \txor operation. As in differential cryptanalysis, approximations of nonlinear components induce a cost in the form of probability. As a result, the resulting equations linking the key, the plaintext and the ciphertext hold only with particular probability. If the adversary observes enough data, then correct equations may be established with high probability. In practice, only the ratio of plaintext-ciphertext pairs for which the equation is correct is computed. For a random permutation, this ratio will be close to 1/2. For a weak block cipher, this ratio may be distinguishable from 1/2 with high probability.

In \emph{integral} cryptanalysis, the \emph{algebraic degree} of a block cipher is studied. It corresponds to the degree of the multivariate polynomial representation of the cipher. If the algebraic degree is not high enough, the cryptanalyst can deduce a set of plaintexts, for which the corresponding ciphertexts \txor to zero, independently of the secret key. Evaluation of the algebraic degree of a cryptographic primitive is a challenging problem and usually, only upper bounds on the degree can be proved. In~\ChapRef{feistel} I describe a method to obtain such upper bounds for the particular block cipher structure, called a \emph{Feistel Network}. It is based on the joint work~\cite{OurFeistel} with my colleague Léo Perrin.

Integral cryptanalysis is one of the main tools for \emph{structural} cryptanalysis. This branch of cryptanalysis studies ways to distinguish \emph{structures} of cryptographic functions and further decompose the function into components of the structure. It means that only the structure of the function is known to the adversary, and its components are kept secret. The most common structures are the substitution-permutation-network (SPN) and the Feistel network (FN). My colleagues Léo Perrin and Alex Biryukov found an intriguing application of structural and decomposition cryptanalysis. They applied it to \emph{small} functions called S-Boxes, which are used to build cryptographic primitives. S-Boxes are usually represented by tables in specifications and the process of their generation may be kept undisclosed. Structural cryptanalysis allows distinguishing particular structures in an S-Box. Together with analysis of resistance against linear and differential attacks, these methods can often reveal secret criteria behind an S-Box of unexplained origin. This direction is called the \emph{reverse-engineering} of S-Boxes. I contributed to the work of my colleagues in reverse-engineering of the S-Box used in the latest Russian cryptographic standards, and reverse-engineering of an S-Box of a mathematical origin. These results are described in \ChapRef{kuz} and \ChapRef{apn} respectively.

A recent direction of cryptanalysis is the search for \emph{invariants} of the cryptographic primitives. \emph{Linear} invariants correspond to a critical flaw in the primitive and are usually easy to avoid. \emph{Nonlinear} invariants are much harder to find. Indeed, the ideas of invariant-based cryptanalysis appeared a long time ago, but the actual applications of the method appeared only recently. A special case of a nonlinear invariant is an \emph{invariant subspace}. Invariant subspace cryptanalysis was introduced in~\cite{InvSpacePrint} and was used to break the PRINTcipher, designed in 2010. Another class of nonlinear invariants is formed by \emph{quadratic} invariants. This class was used in~\cite{NonlinInv} to show a practical distinguisher of recently designed block ciphers Midori, SCREAM, and iSCREAM. In~\PartRef{ni} I describe invariant subspaces in NORX, a CAESAR third round candidate; I also show a theoretical study of generalization of quadratic invariants to higher degrees. This part is based on joint work~\cite{OurNORX} with Alex Biryukov and Vesselin Velichkov, and on joint work~\cite{OurNLI} with Christof Beierle and Alex Biryukov.