\section{Thesis Overview}

In this section, I provide a brief overview of this dissertation. The introduction, the thesis overview, and the list of publications are given in~\textbf{\ChapRef{intro}} (this chapter). \textbf{\ChapRef{prelim}} introduces definitions and notations, together with well-established facts about mathematical structures that are used throughout the thesis. The rest of the work is split into four parts. Each part corresponds to a separate research area that I contributed to during my doctoral studies.


\subsection{\PartRef{str}. Structural and Decomposition Cryptanalysis}

In this part, I present my contribution to structural and decomposition cryptanalysis and S-Box reverse-engineering. It consists of three chapters.

\textbf{\ChapRef{feistel}} shows an application of structural cryptanalysis to the Feistel Network structure. Our research emerged from observing interesting patterns in the linear approximation table of 4- and 5-round Feistel networks with tiny block size. The linear approximation table is used to measure the resistance of the structure to linear cryptanalysis (see Section~\ref{prelim.sec:lindiff} of \ChapRef{prelim}). By studying the artifacts, we deduced and proved their relation to integral cryptanalysis. Further, we generalized and proved these integral properties of Feistel networks. As a result, we obtained a simple closed formula on the number of rounds of Feistel networks, when the integral distinguisher is possible. We represented the integral distinguishers in the form of the \emph{high-degree indicator matrix} (HDIM). Further, we showed the usefulness of HDIM as a tool by cryptanalyzing Feistel networks composed with random affine layers. In addition, we proposed a decomposition attack on Feistel networks based on the integral distinguishers.

\textbf{\ChapRef{kuz}} describes decompositions of the S-Box used in the recent Russian cryptographic standard. With my coauthors, we discovered interesting and unique structures of the S-Box based on the finite field arithmetic. In the chapter, I describe the step-by-step decomposition process. The methods developed in this work will prove their usefulness in~\ChapRef{apn}.

\textbf{\ChapRef{apn}} shows the decomposition of the only known APN permutation in even dimension. APN stands for \emph{almost perfect nonlinear} and corresponds to optimal resistance against differential cryptanalysis. The existence of APN permutations in even dimensions was a long-standing problem until the 6-bit APN permutation was published by Dillon~\etal in 2009. Since then, no new significant progress on the problem was achieved. Furthermore, the method that was used to find the S-Box does not provide any insight on how to generalize the APN permutation. With Alex Biryukov and Léo Perrin, we applied the methods of S-Box reverse-engineering to this S-Box of mathematical origin and surprisingly discovered an interesting structure, which we called a \emph{butterfly}. We studied its properties and generalized it to higher dimensions. Even though we did not find any new APN permutation in even dimension, the generalized butterfly is only slightly weaker than APN permutations.


\subsection{\PartRef{ni}. Nonlinear Invariants}

In this part, I describe my contribution to the method of cryptanalysis based on nonlinear invariants. It consists of two chapters.

\textbf{\ChapRef{norx}} describes an analysis of the core permutation of the NORX authenticated encryption scheme, a third round CAESAR candidate. First, I describe invariant subspaces of the permutation obtained from rotational symmetry of the structure. Second, I show probabilistic invariant subspaces obtained from rotational word symmetry. To illustrate the dangers of such properties, I describe two attacks on slightly modified variants of NORX. Further, I provide the cycle decomposition of a 32-bit mapping $G$ used in the NORX8 instance. I propose an algorithm for the search of low-degree non-linear invariants from a cycle decomposition and apply it to $G$. The results show that there are no low-degree invariants of $G$ holding with probability one. This chapter is based on the joint work~\cite{OurNORX} with Alex Biryukov and Vesselin Velichkov, which is currently available as an online report.

\textbf{\ChapRef{linear}} shows a theoretical study of linear layers that preserve a particular class of low-degree invariants. It is a generalization of the theorem from~\cite{NonlinInv}, where it was shown that orthogonal linear layers preserve a particular class of quadratic invariants. Our study shows that no \emph{bijective} linear layers preserve a similar class of cubic invariants. However, there are such \emph{expanding} linear layers. The linear layers that are studied, correspond to subsets of $\field{n}$ on which every Boolean function of algebraic degree at most $d$, sums to zero. Furthermore, these sets of vectors must have full rank. We call them degree-$d$ zero-sum sets of full rank. The rest of the chapter is devoted to studying the minimum possible size of such sets. This size is related to the minimum expansion rate of the corresponding linear layers. I describe several nontrivial bounds for the minimum possible size of full-rank degree-$d$ zero-sum sets. This chapter is based on the joint work~\cite{OurNLI} with Christof Beierle and Alex Biryukov, which is in the process of submission to a Boolean function journal.


\subsection{\PartRef{wb}. White-box Cryptography}

In this part, I describe my contribution to symmetric cryptography in the white-box model. This part is based on the joint work~\cite{OurWhitebox} with Alex Biryukov. It consists of two chapters. 

\textbf{\ChapRef{wba}} describes several attacks on white-box implementations using masking schemes. Recently, Bos~\etal~\cite{AttackBos} showed that most public white-box implementations can be broken in an automated way by a classic side-channel attack.
It is therefore natural to apply side-channel protection - masking - to protect white-box implementations. However, this chapter shows many caveats that appear in the white-box setting. The described attacks result in constraints that a secure white-box implementation based on masking has to satisfy. In particular, the classic Boolean masking of any order is not secure since it is a linear scheme. 

\textbf{\ChapRef{wbc}} describes a general methodology for securing a white-box implementation. The attacks are split into two groups, and each group corresponds to a separate component of protection. The two components are called \emph{structure hiding} and \emph{value hiding}. Structure hiding protection must hide structural patterns and prevent locating of critical computation points in the circuit by graph-based analysis. It also should include protection against fault attacks, though this may be considered as a separate component. Value hiding protection must prevent  attacks based on analysis of values computed in the white-box circuit, such as side-channel power analysis attacks. In our research, we focused on the value hiding protection, in particular on the novel attack against Boolean masking. We develop a security model and a game-based security definition. Further, we develop a framework of provable security against the attack. Finally, we propose a novel quadratic masking scheme instantiating the developed framework of provable security. We implement AES-128 encryption protected by the novel masking scheme together with the classic Boolean masking scheme to estimate the overhead.


\subsection{\PartRef{de}. Symmetric Algorithm Design}

In this part, I describe my contribution to the design of symmetric-key algorithms. It consists of two chapters.

\textbf{\ChapRef{sparx}} describes SPARX, a lightweight block cipher designed by my colleagues and me. The cipher follows a novel design strategy called a \emph{long-trail strategy}. This strategy provides provable security against single-trail linear and differential cryptanalysis for ARX-based structures, where the classic \emph{wide-trail} strategy is not efficient. I developed two algorithms for long-trail evaluation of a given structure. We evaluated a large class of linear layer structures for the block cipher. During the evaluation, my algorithms were used in order to measure the resistance of the linear layer against linear and differential attacks. I also evaluated the linear layer candidates for resistance against integral attacks using the division property, a state-of-the-art technique. The final choice was done by finding an optimal ratio between the two parameters, and also considering implementation properties. It turned out that a linear Feistel round leads to the best compromise between the parameters. This chapter is based on the joint work~\cite{OurSPARX} with Daniel Dinu, Léo Perrin, Vesselin Velichkov, Johann Großschädl, and Alex Biryukov.

\textbf{\ChapRef{sparkle}} describes a suite of symmetric-key algorithms. SPARKLE is a family of three cryptographic permutations motivated by the SPARX design. ESCH is a family of two hash functions built using the sponge construction and the SPARKLE permutations. SCHWAEMM is a family of authenticated encryption algorithms, built using the recent Bettle sponge-based mode and the SPARKLE permutations. I performed various analyses of the SPARKLE permutation and its components, including nonlinear invariant analysis, a linearization study of the ARX-based S-Box, evaluation of resistance against integral attacks. Furthermore, I propose a generic algorithm for building the matrix of transitions of truncated differential trails through the linear layer. The algorithm takes as input the binary matrix of the linear layer and computes precise probabilities. The advantage of this method is that it automatically takes into account all dependencies between computations in the linear layer, helping to avoid possible mistakes. Finally, I propose several attacks on reduced-round versions of SCHWAEMM. The suite of primitives is submitted to the NIST Call for Lightweight Cryptographic Algorithms~\cite{NISTlight}. It is a joint work with Christof Beierle, Alex Biryukov, Luan Cardoso dos Santos, Johann Großschädl, Léo Perrin, Vesselin Velichkov and Qingju Wang.