\SecDef{intro}{Introduction}

In the traditional symmetric cryptography, an adversary has access only to the inputs and outputs of a cryptographic primitive. This model is called the \emph{black-box} model. Relaxation of this model is called \emph{grey-box} and in it attacker may also obtain side-channel or fault information from the cryptographic implementation.  
In the extreme \emph{white-box} model the adversary is given full access to the implementation which contains secret keys. He can use both static and dynamic analysis as well as fault analysis in order to break the cryptosystem, e.g. to extract embedded secret keys. Implementations secure in such model have many applications in industry. However, creating such implementations turns out to be a very challenging if not an impossible task.

In 2002, Chow \etal{}~\cite{ChowAES,ChowDES} proposed first white-box implementations of the AES and DES block ciphers. The main idea is to represent small parts of a block cipher as look-up tables and compose them with randomized invertible mappings to hide the secret key information. Each such look-up table by itself does not give any information about the key. In order to attack such scheme, multiple tables must be considered. Another approach was proposed by Bringer \etal{}~\cite{cryptoeprint:2006:468}. Instead of look-up tables, the cipher is represented as a sequence of functions over $\fielde{n}$ for some $n$, with some additional computations as noise. These functions are then composed with random linear mappings to hide the secret key, similarly to the Chow \etal{} approach.

Unfortunately, both approaches fell to practical attacks~\cite{AttackBillet,AttackMulder,AttackLepoint}. Consequent attempts to fix them were not successful~\cite{Dual,Xiao}. Moreover, Michiels~\etal{}~\cite{AttackMichiels} generalized the attack by Billet \etal{}~\cite{AttackBillet} and showed that the approach of Chow~\etal{} is not secure for any SPN cipher with MDS matrices. This follows from the efficient cryptanalysis of any SASAS structure~\cite{SASAS}. Recently several white-box schemes based on the ASASA structure were proposed~\cite{ASASAdesign}. However the strong white-box scheme from that paper was broken~\cite{ASASA2,GPT15,BKP17} (which also broadens the white-box attacker's arsenal even further). Another recent approach consists in obfuscating a block cipher implementation using candidates for indistinguishability obfuscation (e.g.~\cite{iOcand}).

Besides academia, there are commercial white-box solutions that are used in real products. The design behind those implementations is kept secret, thus adding \emph{security-by-obscurity} protection. Nevertheless, Bos \etal{} \cite{AttackBos} proposed a framework for attacks on white-box implementations which can automatically break many white-box implementations. The idea is to apply techniques from grey-box analysis (i.e. side-channel attacks) but using more precise data traces obtained from the implementation. The attack is called \emph{differential computation analysis} (DCA). Sasdrich~\etal{}~\cite{Sasdrich} pointed out that the weakness against the DCA attack can be explained using the Walsh transform of the encoding functions. Banik~\etal{}~\cite{Subhadeep} analyzed software countermeasures against the DCA attack and proposed another automated attack called Zero Difference Enumeration attack. More recently, Bock~\etal{}~\cite{InternalEncodings} analyzed internal encodings in white-box implementations. Consequently, Rivain and Wang~\cite{WangEncodings} provided in-depth analysis and showed that internal encodings can be easily broken in most cases, improved the attack complexities and proposed a new collision attack.

In light of such powerful automated attack the question arises: how to create a whitebox scheme secure against the DCA attack? The most common countermeasure against side-channel attacks is masking, which is a form of secret sharing. It is therefore natural to apply masking to protect white-box implementations. We define masking to be any obfuscation method that encodes each original bit by a relatively small amount of bits. Such masking-based obfuscation may be more practical in contrast to cryptographic obfuscation built from current indistinguishability obfuscation candidates~\cite{iOcand,5GEN}. 

\subsection{Our Contribution}

This chapter studies the possibility of using masking schemes in the white-box setting. We restrict the analysis to implementations in the form of Boolean circuits. 

We develop a more generic DCA framework and describe multiple generic attacks against masked implementations. The attacks show that the classic Boolean masking (\txor{}-sharing) is inherently weak. Previous and new attacks are summarized in \TabRef{attacks}. We remark that conditions for different attacks vary significantly and the attacks should not be compared solely by time complexity. For example, the fault-based attacks are quite powerful, but it is relatively easy to protect an implementation from these attacks. From the attacks we conclude that more general nonlinear encodings are needed and we deduce constraints that a secure implementation must satisfy. We believe that these results provide new insights on the design of white-box implementations. Note that a basic variant of the (generalized) linear algebra attack was independently proposed by Goubin~\etal{}~\cite{cryptoexperts}.

\FigTex{attacks.tex}

A code implementing the described attacks and protections from \ChapRef{wbc} is publicly available at~\cite{OurWhiteboxCode}:
\begin{center}
    \url{https://github.com/cryptolu/whitebox}
\end{center}

\subsection{Outline}
The general attack setting and attacks are described in \SecRef{dca}. Combinatorial and algebraic attacks in the DCA setting are described in \SecRef{combinatorial} and \SecRef{algebraic} respectively. In \SecRef{fault} I suggest fault-based attacks.
Finally, I conclude in \SecRef{conclusions}.