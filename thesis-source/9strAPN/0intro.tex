\SecDef{intro}{Introduction}

S-Boxes are used to provide non-linearity in SPN-based block ciphers. They provide basic resistance against linear and differential cryptanalysis, and the rest of the structure ensures that many S-Boxes are activated in a linear or differential trail. The resistance of an S-Box can be quantified. The lower are the \emph{linearity} and the \emph{differential uniformity} of an S-Box, the more resistant it is. Everything else being equal, a stronger S-Box allows to use less rounds in the block cipher using it for the same security level.

The differential uniformity is always even and at least equal to 2. When this bound is achieved, the S-Box is called \emph{Almost Perfect Non-linear (APN)}. The finite field cube function is APN in all field dimensions~\cite{Nyb94}. However, it is a permutation only in odd dimensions. This is a problem, since in most cases (e.g. an SPN block cipher) the S-Boxes are required to be bijective. For efficiency reasons, even-dimensional S-Boxes are preferable, especially powers of 2. And this is exactly the case, where the existence of APN functions is not established: bijective S-Boxes in even dimensions with differential uniformity 2, i.e. APN permutations of $\field{n}$ for $n$ even. For $n = 4$ there exist no APN permutations of $\field{n}$. For $n = 6$ this question was a long standing problem until Dillon~\etal{} presented a 6-bit APN permutation~\cite{DillonAPN,DillonPres} in 2009. Since then, no answers were obtained for even $n \ge 8$, despite many attempts~\cite{SubspaceProperty,Uniform4}. This remains a big open problem in the field of Boolean functions.

The 6-bit APN permutation is found by a computer search, by transforming the 6-bit APN function, called the Kim mapping $\kim\colon \fielde{6} \to \fielde{6}$:
$$
\kim(x) \eqdef \VV x^{24} + x^{10} + x^3,
$$
where $v$ is a primitive element of $\fielde{6}$. Even though the Kim mapping is a trinomial function, the resulting APN permutation is an object without clear structure. For example, its polynomial form contains 52 monomials.

Using the methods of S-Box reverse-engineering described in previous chapters (developed in~\cite{OurKuz1}), I and my coauthors managed to find a simple algebraic structure of the Dillon's APN permutation. We call this structure a ``Butterfly'' because of its graphical representation the way it changes by particular transformations. The decomposition is established in Theorem~\Ref{thm:main}, restated here:

\textbf{Main theorem (A Family of 6-bit APN Permutations)}.
The 6-bit permutation described by Dillon \etal{} in~\cite{DillonAPN} is affine equivalent to any involution built using the structure described in \FigRef{intro-decomp}, where $\fmult$ denotes multiplication in the finite field $\fielde{3}$, $\alpha \neq 0$ is such that $\tr(\alpha) = 0$ and $\mathcal{A}$ denotes any 3-bit APN permutation.
\FigTex{intro-decomp.tex}

\subsection{Notations}
For any $f\colon \field{n} \to \field{n}$ let $\parf{f}\colon \field{2n} \to \field{2n}$ be the parallel application of $f$ given by
$$
\parf{f}(x,y) = (f(x), f(y)).
$$
For any $a,b \in \field{n}$ let $\xorf{f}\colon \field{2n} \to \field{2n}$ be parallel xor with constants $a,b$:
$$
\xorf{f}(x,y) = (x \oplus a, y \oplus b).
$$
The finite field trace function is denoted by $\Tr\colon \fielde{n} \to \field{}$, it is given by
$$
\Tr(x) \eqdef \sum_{e=0}^{n-1} x^e.
$$

\subsection{Outline}
\SecRef{decomposition} explains the decomposition process of the APN permutation. In~\SecRef{properties} I describe new properties of the APN permutation that follow from the discovered structure. \SecRef{components} studies the flexibility of the structure, i.e. how can we modify the structure while preserving the APN property?
In \SecRef{relations} I show new relations between the APN permutation, the Kim mapping, monomial functions and 3-round Feistel Network structure. Finally, I briefly conclude the chapter in \SecRef{conclusions}.
