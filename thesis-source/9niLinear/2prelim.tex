\SecDef{prelim}{Preliminaries}

In this chapter I use a specific notation to simplify expressions. A vector $v \in \field{n}$ is considered a row vector. Any matrix $L \in \F_2^{n\times m}$ defines a linear mapping $\varphi \colon \F_2^n \rightarrow \F_2^m, \ x \mapsto xL$.

\subsection{Higher-Order Derivatives, Affine Equivalence and Algebraic Immunity of Boolean Functions}

Boolean functions have several applications in cryptography, e.g., for designing stream ciphers. In order to resist algebraic attacks, the notion of algebraic immunity was introduced in 2004 as follows.

\begin{definition}[Algebraic immunity~\cite{EC:MeiPasCar04}]
Let $f\colon \F_2^n \rightarrow \F_2$. An $n$-bit Boolean function $g \neq 0$ is called an \emph{annihilator} of $f$, if $f g = 0$. The set of annihilators of $f$ together with $g=0$ form a vector space, denoted by $\AN(f)$. We denote by $\AN_d(f)$ the subspace of annihilators of $f$ with algebraic degree at most $d$ together with the zero-function. The \emph{algebraic immunity} of $f$, denoted $\AI(f)$, is defined as the minimum $k$ for which $\AN_k(f) \cup \AN_k(f+1) \neq \{0\}$.
\end{definition}

An important concept for Boolean function is the notion of affine equivalence. 

\begin{definition}[Domain Affine Equivalence]
Two Boolean functions $f,g\colon \F_2^n \rightarrow \F_2$ are called \emph{domain affine equivalent} if there exists a linear bijection $\varphi \colon \F_2^n \rightarrow \F_2^n$ and a vector $c \in \F_2^n$ such that $g = f \circ (\varphi + c)$. If $c = 0$, $f$ and $g$ are called \emph{linear equivalent}.
\end{definition}
I remark that, in the literature, \emph{domain affine equivalence} of Boolean functions is called simply \emph{affine equivalence}. I specify the term to avoid ambiguity, as, for example, $g$ and $g\oplus1$ are not domain affine equivalent in general.
It is well known that the weight, the algebraic degree and the dimensions of the annihilator spaces (and thus the algebraic immunity) are invariant under domain affine equivalence. 

\subsection{Orthogonal Matrices and Preservation of Nonlinear Invariants}

In~\cite{NonlinInv}, Todo, Leander and Sasaki introduced the \emph{nonlinear invariant attack} and successfully distinguished the block ciphers Midori, Scream and iScream from a random permutation for a significant fraction of weak keys. For an $n$-bit permutation $G \colon \F_2^n \rightarrow \F_2^n$, the main idea consists in finding a non-constant $n$-bit Boolean function $f$ and a constant $\varepsilon \in \F_2$ such that 
$$
\forall x \in \F_2^n\colon f(x) = f(G(x)) + \varepsilon.
$$

Such a function $f$ is called an \emph{invariant} for $G$.
In order to find an invariant for the cipher, Todo \etal. observed that if $L \in \F_2^{n \times n}$ is an orthogonal matrix, i.e., if $\inprod{xL, yL} = \inprod{x, y}$ for all $x,y \in \F_2^n$, then for all Boolean functions $f \in \BF{t}{2}$ it is
\eql{target_equation}{
\forall X \in \F_2^{t \times n} \colon
\bigoplus_{i=1}^n f\big( (X^\top)_i \big) =
\bigoplus_{j=1}^n f\big( ((XL)^\top)_j \big).
}
In other words, \emph{any} Boolean function $f\colon \F_2^t \rightarrow \F_2$ of algebraic degree at most 2 gives rise to an invariant over the linear layers of Midori, Scream and iScream of the form
$(x_1,\dots,x_n) \mapsto f(x_1)+\dots f(x_n)$, where $n$ denotes the number of S-boxes, $t$ denotes the bit length of the S-box and $x_i \in \F_2^t$.
 
We illustrate this from a slightly different point of view on the example of the linear layer used in Midori (see~\cite{AC:BBISHA15}), which is defined by the following matrix:
\eql{midori}{
L = \matb{
0 & 1 & 1 & 1 \\
1 & 0 & 1 & 1 \\
1 & 1 & 0 & 1 \\
1 & 1 & 1 & 0
}.
}
It is easy to see that $L$ is orthogonal. Thus, according to \EqRef{target_equation}, for \emph{any} $f \in \BF{t}{2}$ and all $x_1,x_2,x_3,x_4 \in \F_2^t$, the following equation holds:
\eq{
    &f(x_1) + f(x_2) + f(x_3) + f(x_4) = \\ &f(x_2+x_3+x_4)+f(x_1+x_3+x_4)+f(x_1+x_2+x_4)+f(x_1+x_2+x_3).
}

Consider an alternative way of proving this. The arguments of $f$ form an affine subspace of dimension 3, namely
$$
x_1 + \Span(x_1+x_2, x_1+x_3, x_1+x_4).
$$
Therefore, the equation is equivalent to
$$
    \delta_{x_1+x_2}\delta_{x_1+x_3}\delta_{x_1+x_4}f(x_1) = 0,
$$
which is clearly true for any $f \in \BF{t}{2}$ and any $x_1,x_2,x_3,x_4$ since all third-order derivatives of a quadratic function are equal to zero. This observation gives new insights on how to generalize the linear layer in order to preserve \emph{higher-degree invariants}.

\begin{proposition}
\PropLabel{expanding-linear-high-degree}
Let $d \ge 2$ be an integer. Then there exists a matrix $L \in \F_2^{n \times m}$ with $n = d+2, m=2^{d+1}-d-2$ and full rank $n$ such that for any $t \ge 1$ and any $f \in \BF{t}{d}$, the following property holds: 
\eql{invariant_preserving_linear}{
\forall X \in \F_2^{t \times n} \colon
\bigoplus_{i=1}^n f\big( (X^\top)_i\big) =
\bigoplus_{j=1}^m f\big( ((XL)^\top)_j \big).
}
An example of such $L$ is given by a matrix with columns taken as all vectors from $\field{n}$ with an odd Hamming weight greater or equal to 3.
\end{proposition}
\begin{proof}
For any $t \ge 1$ and any $x_0,\ldots,x_{d+1}\in \F_2^t$ consider the $(d+1)$-dimensional affine subspace
\[V = x_0 + \Span(x_0+x_1,x_0+x_2,\ldots,x_0+x_{d+1})\;.\]
For any Boolean function $f$ of degree $d$, any $(d+1)$-th derivative vanishes. Therefore, $\bigoplus_{v \in V} f(v) = 0$. This can be equivalently written as
\eql{higher-order-expanding}{
    & f(x_0) + f(x_1) + \ldots + f(x_{d+1}) = \\
    & = \bigoplus_{\substack{I \subseteq \{1, \ldots, d+1\} \\ |I| \ge 2 ~\text{even}}}f(x_0 + \bigoplus_{i \in I}x_i) +
        \bigoplus_{\substack{I \subseteq \{1, \ldots, d+1\} \\ |I| \ge 3 ~\text{odd}}}f(\bigoplus_{i \in I}x_i) \\
    & = \bigoplus_{\substack{I \subseteq \{0, \ldots, d+1\} \\ |I| \ge 3 ~\text{odd}}}f(\bigoplus_{i \in I}x_i)
    .
}
The right-hand side contains $2^{d+1}-d-2$ applications of $f$. Let $Y$ be the set of the linear functions defining the arguments of $f$ in the right-hand side of \EqRef{higher-order-expanding}, i.e.,
$$
Y =
\pset{ \bigoplus_{i \in I}x_i ~\biggm|~ I \subseteq \{0, \ldots, d+1\}, \ |I| \ge 3 ~\text{odd} },
$$
and let $L$ be the matrix of the linear function that maps $(x_0, x_1, \ldots, x_{d+1})$ to $(y_1, y_2, \ldots, y_{2^{d+1}-d-2})$, where $y_i \in Y$ and all $y_i$ are pairwise different. Then, \EqRef{higher-order-expanding} is equivalent to \EqRef{target_equation} with the described $L$.

Since $m \ge n \ge 4$, any unit vector from $\F_2^n$ can be expressed a linear combination of 3 columns of $L$, e.g., $$
(1,0,0,0,\ldots,0)=(1,1,1,0,\ldots,0)+(1,0,1,1,\ldots, 0)+(1,1,0,1,\ldots,0).
$$
We conclude that $L$ has full rank $n$.
\end{proof}


\begin{example}
For $d=2$ we obtain the orthogonal matrix given in \EqRef{midori}. For $d=3$ we obtain an expanding linear mapping $\varphi: \field{5} \to \field{11}$ defined by the following $5 \times 11$ matrix $L$:
$$
L = \matb{
0 & 0 & 0 & 0 & 1 & 1 & 1 & 1 & 1 & 1 & 1 \\
0 & 1 & 1 & 1 & 0 & 0 & 0 & 1 & 1 & 1 & 1 \\
1 & 0 & 1 & 1 & 0 & 1 & 1 & 0 & 0 & 1 & 1 \\
1 & 1 & 0 & 1 & 1 & 0 & 1 & 0 & 1 & 0 & 1 \\
1 & 1 & 1 & 0 & 1 & 1 & 0 & 1 & 0 & 0 & 1 \\
}.
$$
\end{example}


