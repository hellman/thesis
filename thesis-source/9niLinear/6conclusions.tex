\SecDef{conclusions}{Conclusion and Open Problems}

In the work I described in this chapter we have revealed the precise properties of the linear layer used in LS-designs that allow to preserve nonlinear invariants of a similar form than those observed by Todo \etal. As a negative result, we have shown that it is not possible to construct such an LS-design block cipher that generalizes the invariants to be preserved up to algebraic degree $3$. Those results were obtained by studying the Boolean functions of minimum weight that admit no linear annihilator.

An interesting open question is stated in Question~\Ref{q:minimal-maximal}. That is, can we understand in which cases the minimal degree-$d$ zero-sum sets are also maximal? A more general and indeed remarkable result would be to derive exact formulas for $\minzs{n}{d}$ in those cases where we were only able to provide upper and lower bounds. Indeed, solutions to those problems would have interesting implications such as understanding the minimum expansion rate of degree-$d$ sum-invariant matrices and deriving equivalences between degree-$d$ zero-sum sets and Boolean functions with algebraic immunity at least $2$.