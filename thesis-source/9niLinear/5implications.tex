\SecDef{implications}{Implications for Degree-\texorpdfstring{$d$}{d} Sum-Invariant Matrices}

In this section, I point out the implications of the above results on degree-$d$ sum-invariant matrices. The most interesting implication is that any bijective degree-$3$ sum-invariant matrix must be trivial. As the linear layer of a block cipher based on an LS-design certainly has to be bijective, this shows that one cannot extend the observation of Todo \etal. to invariants of degree higher than two.

\begin{corollary}
Let $L \in \F_2^{n \times n}$ be a degree-$d$ sum-invariant matrix for $d \geq 3$. Then $L$ must be a permutation matrix.
\end{corollary}
\begin{proof}
Let us assume a degree-$3$ sum-invariant matrix $L \in \F_2^{n \times n}$ and let $\matl{L}$ be given by
\[
\arraycolsep=4pt\def\arraystretch{1}
    \matl{L} = \left[\begin{array}{c|c} \idmat{n} & L\end{array}\right] \in \F_2^{n\times 2n}\;.
\]
By \PropRef{inner_product} the columns of $\matl{L}$ occurring an odd number of times correspond to a degree-3 zero-sum set $S \subseteq \F_2^n$. Note that the unit columns of $\idmat{n}$ do not repeat inside $\idmat{n}$. Therefore, after removing the even occurrences of each column, the number of columns left in $\idmat{n}$ will be not smaller than the number of columns left in $L$. It follows that $\rank(S) \ge |S|/2$. From Corollary\Ref{cor:Fn3},
$$
|S|\ge F(\rank(S), 3) \ge 2 \cdot \rank(S) + 6
$$
Therefore, $S$ must be empty and thus $L$ is a permutation matrix.
\end{proof}

Consider a degree-$d$ sum-invariant matrix $L$ and consider the matrix $\matl{L}$ defined as in \PropRef{inner_product}:
    \begin{equation}
    \begin{cases}
    \arraycolsep=4pt\def\arraystretch{1}
    \matl{L} \coloneqq \left[\begin{array}{c|c} \idmat{n} & L\end{array}\right] \in \F_2^{n\times(m+n)},
    & \text{~if~} m+n \text{~is even}; \\
    \arraycolsep=4pt\def\arraystretch{1}
    \matl{L} \coloneqq \left[\begin{array}{c|c|c} \idmat{n} & L & 0\end{array}\right] \in \F_2^{n\times(m+n+1)},
    & \text{~if~} m+n \text{~is odd},
    \end{cases}
    \end{equation}
where it is shown that the columns of $\matl{L}$ occurring and odd number of times define a degree-$d$ zero-sum set. Because of the cancellations, the size and the rank of the zero-sum set may be lower. We deduce the following decomposition of sum-invariant matrices.

\begin{proposition}
Let $L \in \F_2^{n \times m}$ be a degree-$d$ sum-invariant matrix such that no column of $L$ is equal to zero. Then, up to permutations of rows and columns, $L$ can be expressed in the following form:

\eql{invariant-matrix-form}{
    L = \left[ \begin{array}{c|c|c|c} 
    A &
    \coltworaw{0}{I_k} &
    M &
    M 
    \end{array} \right],
}
where $k,t$ are some integers, $M \in \F_2^{n \times t}$, $A \in \F_2^{n \times (m - 2t - k)}$, and the columns of $A$ do neither contain unit vectors nor repetitive columns. Such integers $k,t$ are unique. 
Consider the matrix $\widehat{A}$:
\begin{equation}
    \begin{cases}
    \arraycolsep=4pt\def\arraystretch{1}
    \widehat{A} \coloneqq\left[
    \begin{array}{c|c} \coltworaw{I_{n-k}}{0} & A\end{array}
    \right] \in \F_2^{n\times(m+n-2t-2k)},
    & \text{~if~} m+n \text{~is even}; \\
    \arraycolsep=4pt\def\arraystretch{1}
    \widehat{A} \coloneqq \left[
    \begin{array}{c|c|c} \coltworaw{I_{n-k}}{0} & A & 0\end{array}\right
    ] \in \F_2^{n\times(m+n-2t-2k+1)},
    & \text{~if~} m+n \text{~is odd}.
    \end{cases}
\end{equation}
The columns of the matrix $\widehat{A}$ are pairwise distinct and form a degree-$d$ zero-sum set.
% Q: rank of ZS is rank of A???
\end{proposition}
\begin{proof}
The columns of $\matl{L}$ occurring an odd number of times form a degree-$d$ zero-sum set. The columns of $\idmat{n}$ may only cancel with columns from $L$. Let $k$ be the number of unit vectors occurring an odd number of times in $L$. Let $A$ be the matrix consisting of the columns of $L$ that are repeated an odd number of times and which are not unit vectors. It follows that $L$ can be expressed in the form given in \EqRef{invariant-matrix-form}. Now consider the matrix $\matl{L}$. After removing even repetitions of columns, the matrix will be equal to $\widehat{A}$. It follows that the columns of $\widehat{A}$ define a degree-$d$ zero-sum set. 

To show uniqueness of $k,t$, first recall that $A$ must not contain unit vectors. It follows that all columns of $L$ occurring an even number of times must be in $M$, and all columns occurring an odd number of times must be either in $A$ or in $I_k$ depending only on the column weight.

\end{proof}

% =====================================

\subsection{Minimum Expansion Rate}
We have shown that for $d \ge 3$, there exist no bijective degree-$d$ sum-invariant matrices. However, there exist rectangular degree-$d$ sum-invariant matrices resulting in expanding linear mappings. 
A natural problem would be to find a degree-$d$ sum-invariant matrix with a minimum expansion rate.



\begin{definition}[Expansion Rate]
The \emph{expansion rate} of a matrix $L \in \F_2^{n\times m}$ is the ratio $\frac{m}{n}$.
\end{definition}

Note that, given a degree-$d$ sum-invariant matrix $L \in \F_2^{n \times m}$, we can always build a a degree-$d$ sum-invariant matrix in $\F_2^{(n+1)\times(m+1)}$ of the form
\[ \left[\begin{array}{cc} L & 0 \\
0 & 1\end{array}\right]\;.\]
Therefore, by repetitively extending any matrix $L$ by unit vectors in the above way, we can construct a matrix with an expansion rate arbitrarily close to $1$. Indeed, the permutation matrices have an expansion rate of exactly $1$. Therefore, by the \emph{minimum expansion rate} for a degree-$d$ sum-invariant matrix of fixed $d$, we refer to the minimum expansion rate over all degree-$d$ sum-invariant matrices that do not contain a unit vector as a column.

It is clear that for $d=2$ the minimum expansion rate is $1$ and is achieved by orthogonal matrices. For $d \ge 3$ the minimum expansion rate is an open problem. It corresponds to the minimum value of $\frac{F(n,d)}{n}-1$. Among the established values of $F(n,d)$ the minimum expansion rate is achieved for $F(d+2,d)=2^{d+1}$, i.e. by the matrices from the construction given in \PropRef{expanding-linear-high-degree}. We conjecture that this is indeed the optimal expansion rate.

\begin{conjecture}
\Label{conj:minimum-expansion}
Let $d \ge 3$. The minimum expansion rate of a degree-$d$ sum-invariant matrix is equal to $\frac{2^{d+1}-d-2}{d+2}$.
\end{conjecture}


