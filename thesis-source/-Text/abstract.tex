\begin{abstract}
\addchaptertocentry{\abstractname} % Add the abstract to the table of contents

Cryptography studies secure communications. In symmetric-key cryptography, the communicating parties have a shared secret key which allows both to encrypt and decrypt messages. The encryption schemes used are very efficient but have no rigorous security proof. In order to design a symmetric-key primitive, one has to ensure that the primitive is secure at least against known attacks. During 4 years of my doctoral studies at the University of Luxembourg under the supervision of Prof. Alex Biryukov, I studied symmetric-key cryptography and contributed to several of its topics.

\PartRef{str} is about the \emph{structural} and \emph{decomposition} cryptanalysis. This type of cryptanalysis aims to exploit properties of the algorithmic structure of a cryptographic function. The first goal is to \emph{distinguish} a function with a particular structure from random, structure-less functions. The second goal is to \emph{recover} components of the structure in order to obtain a \emph{decomposition} of the function. Decomposition attacks are also used to uncover secret structures of S-Boxes, cryptographic functions over small domains. In this part, I describe structural and decomposition cryptanalysis of the Feistel Network structure, decompositions of the S-Box used in the recent Russian cryptographic standard, and a decomposition of the only known APN permutation in even dimension.

\PartRef{ni} is about the \emph{invariant}-based cryptanalysis. This method became recently an active research topic. It happened mainly due to recent ``extreme'' cryptographic designs, which turned out to be vulnerable to this cryptanalysis method. In this part, I describe an invariant-based analysis of NORX, an authenticated cipher. Further, I show a theoretical study of linear layers that preserve low-degree invariants of a particular form used in the recent attacks on block ciphers.

\PartRef{wb} is about the \emph{white-box} cryptography. In the white-box model, an adversary has full access to the cryptographic implementation, which in particular may contain a secret key. The possibility of creating implementations of symmetric-key primitives secure in this model is a long-standing open question. Such implementations have many applications in industry; in particular, in mobile payment systems. In this part, I study the possibility of applying \emph{masking}, a side-channel countermeasure, to protect white-box implementations. I describe several attacks on direct application of masking and provide a provably-secure countermeasure against a strong class of the attacks.

\PartRef{de} is about the \emph{design} of symmetric-key primitives. I contributed to design of the block cipher family SPARX and to the design of a suite of cryptographic algorithms, which includes the cryptographic permutation family SPARKLE, the cryptographic hash function family ESCH, and the authenticated encryption family SCHWAEMM. In this part, I describe the security analysis that I made for these designs.
\end{abstract}