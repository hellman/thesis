\SecDef{conclusions}{Conclusions}

In this chapter we investigated the possibility of using masking techniques for white-box implementations. We presented several attacks applicable in different scenarios. As a result we obtained requirements for a masking scheme to be useful. We divided the requirements into \VH{} and \SH{} protections. Furthermore, we suggested that \VH{} may be achieved using an algebraically secure nonlinear masking scheme and a classic linear masking scheme. We developed a framework for provable security against the algebraic attack and proposed a concrete provably secure first-order masking scheme. Therefore, a \VH{} protection can be implemented. 

We believe that our work opens new promising directions in obfuscation and white-box design. We focused on \VH{} protection and developed a first-order protection against the algebraic attack. The natural open question is developing higher-order countermeasures for the algebraic attack. Another direction is to study \SH{} countermeasures. Finally, it seems that pseudorandom generators play an important role in white-box obfuscation and are useful at all layers of protection. Randomness helps to develop formal security models and pseudorandom generators bridge the gap between theoretical constructions and real world implementations. Therefore, designing an easy-to-obfuscate pseudorandom generators is another important open problem.