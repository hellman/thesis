\SubSecDef{construction}{First-order Secure Construction}

Given the notion of prediction security we are now interested in developing secure constructions. A common strategy is to develop small secure circuits (called \emph{gadgets}) and compose them in a provably secure way. The definition of prediction security does not immediately lead to composability, because it includes the encoding step which is not expected to be present in the intermediate gadgets. In order to proceed, we split up the prediction security into circuit security and encoding security. The new notions are stronger in order to get proofs of secure composability. They are limited to the first-order security ($d=1$) and it is not obvious how to extend them to higher orders.

\begin{definition}[Circuit Algebraic Security ($\maxcor$-1-AS)]$ $\newline
Let $C(x,r): \field{N'} \times \field{\rC} \to \field{M}$ be a Boolean circuit. Then $C$ is called first-order algebraically $\maxcor$-secure ($\maxcor$-1-AS) if for any $f \in \FUNCSD{1}(C)\notconst$ one of the following conditions holds:
\begin{enumerate}
    \item $f$ is an affine function of $x$,
    \item for any $x \in \field{N'}$, $|\cor(f(x, \cdot))| \le \maxcor$, where $f(x, \cdot): \field{\rC} \to \field{}$.
\end{enumerate}
\end{definition}
 
\begin{definition}[Encoding Algebraic Security ($\maxcor$-1-AS)]$ $\newline
Let $\Enc(x,r): \field{N} \times \field{\rE} \to \field{N'}$ be an arbitrary encoding function. Let $\mathcal{Y}$ be the set of the coordinate functions of $\Enc$ (i.e. functions given by the outputs bits of $\Enc$).
The function $\Enc$ is called a first-order algebraically $\maxcor$-secure encoding ($\maxcor$-1-AS) if for any $f \in \mathcal{Y}^{(1)}\notconst$ and for any $x \in \field{N}$,
$$
    |\cor(f(x, \cdot))| \le \maxcor,
$$
where $f(x, \cdot): \field{\rE} \to \field{}$.
\end{definition}

The following proposition shows that if both an encoding and a circuit are algebraically secure, then their combination is prediction-secure:

\begin{proposition}
\PropLabel{combine}
Let $C(x',r): \field{N'} \times \field{\rC} \to \field{M}$ be a Boolean circuit and
let $\Enc(x,r): \field{N} \times \field{\rE} \to \field{N'}$ be an arbitrary encoding function.

If $C$ is $\maxcor_C$-1-AS circuit and $\Enc$ is $\maxcor_E$-1-AS encoding, then the pair $(C, E)$ is a $\max(\maxcor_C,\maxcor_E)$-1-AS scheme.
\end{proposition}
\begin{proof}
    If the function $\tilde{f}$ chosen by the adversary is an affine combination of the input $x'$ of $C$, then the encoding security of $\Enc$ applies leading to the bound with $\maxcor=\maxcor_E$. Otherwise, $\maxcor_C$-1-AS security of $C$ provides the bound with $\maxcor=\maxcor_C$ (the $\maxcor_C$ bound applies for any fixed input $x'$ of $C$, therefore it applies for any distribution of $x'$ generated by $E$ as well).
\end{proof}

Finally, we show that $\maxcor$-1-AS circuits are composable, i.e. are secure gadgets. We can compose gadgets in arbitrary ways and then join the final circuit with a secure encoding function to obtain a prediction-secure construction.

\begin{proposition}[$\maxcor$-1-AS Composability]
\PropLabel{composability}
Consider $\maxcor$-1-AS circuits $C_1(x_1, r_1)$ and $C_2(x_2, r_2)$. Let $C$ be the circuit obtained by connecting the output of $C_1$ to the input $x_2$ of $C_2$ and letting the input $r_2$ of $C_2$ be the extra input of $C$:

$$
C(x_1, (r_1, r_2)) \eqdef C_2(C_1(x_1, r_1), r_2).
$$

Then $C(x_1, (r_1, r_2))$ is also a $\maxcor$-1-AS circuit.
\end{proposition}
\begin{proof}
Consider an arbitrary function $\tilde{f}(x_1,r_1,r_2) \in \FUNCSD{1}(C)$. By linearity, it can be written as $u \oplus v$, where $u \in \FUNCSD{1}(C_1)$ and $v$ is a function from $\FUNCSD{1}(C_2)$ composed with $C_1$ (by connecting the output of $C_1$ to the input $x_2$ of $C_2$). Since $C_2$ is $\maxcor$-1-AS, $v$ is either an affine function of $x_2$ (which belongs to $\FUNCSD{1}(C_1)$) or $|\cor(v)|$ is not greater than $\maxcor$ when $x_2$ is fixed (i.e. when $x_1,r_1$ are fixed). In the first case, we get that $\tilde{f}$ belongs to $\FUNCSD{1}(C_1)$ and security follows from $\maxcor$-1-AS security of $C_1$. In the second case, observe that the absolute  correlation of $v$ can not exceed $\maxcor$ for any fixed $x_2$ and, therefore, it can not exceed $\maxcor$ for any distribution of $x_2$. Moreover, $u$ is independent from $r_2$. Therefore, for $\tilde{f} = u \oplus v$ it follows
that $|\cor(\tilde{f})| \le |\cor(v)| \le \maxcor$ since $C_2$ is an $\maxcor$-1-AS circuit.
\end{proof}

This result shows that due to frequent use of fresh randomness it is guaranteed that the maximum bias does not grow when we build large algebraically secure circuits from smaller ones. It means that $\maxcor$-1-AS circuits offer a solid protection against the LPN-based variant of the algebraic attack as well. The complexity of LPN algorithms grows exponentially with the number of unknowns. Therefore, increasing the number of random nodes as suggested by the Corollary~\Ref{cor:seclevel} allows to reach any required level of security against LPN attacks at the same time. Exact required number of random nodes depends on the value of $\maxcor$ and chosen LPN algorithm.