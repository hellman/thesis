In this part, I present the work I have done on cryptanalysis techniques based on nonlinear invariants and, in particular, invariant subspaces. These techniques were applied recently to several lightweight block ciphers~\cite{InvSpacePrint,NonlinInv} and are studied actively~\cite{NIproof,eigenNI,genNI,NIrevisited}.

First, I contributed to the analysis of the permutation used in the NORX authenticated encryption, a third-round candidate of the CAESAR competition~\cite{CAESAR}. It is a joint work with Alex Biryukov and Vesselin Velichkov, available as a report~\cite{OurNORX}.
We found symmetries on different levels of the structure and verified the absence of nonlinear invariants of low degree in the $G$ function used in NORX8. One of the symmetries was independently discovered in~\cite{NORXfse} and was used to attack the previous version of NORX.

Second, together with Christof Beierle and Alex Biryukov, we developed theoretical aspects of nonlinear invariants with respect to linear layers~\cite{OurNLI}. We studied whether quadratic invariant attacks from ~\cite{NonlinInv} can be generalized to higher degrees. As one of our results, we prove that there exist no bijective linear maps that preserve cubic invariants of the same form as in~\cite{NonlinInv}. We also show that such \emph{expanding} linear maps exist and study the minimum expansion rate. This work is currently available as a report~\cite{OurNLI}.