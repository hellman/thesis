In this part, I present the work I have done on structural and decomposition attacks. These attacks aim at determining or distinguishing a particular structure of a cryptographic primitive, which is typically provided as an oracle. Once a structure is established, the cryptanalyst tries to recover its components. For example, the structure may be a Feistel Network with the Feistel round functions being the components. Alternatively, a Substitution-Permutation-Network may be analyzed, where the components are the S-Boxes and the affine mixing layers. There are several reasons for studying structural and decomposition attacks.

First, one may consider a cryptographic primitive, for example, a block cipher, with a publicly known structure but with secret components. For instance, one may replace the components of the AES block cipher - the S-Boxes and the affine layers - by secret ones (see~\cite{AESsecret1,AESsecret2}). The description of the secret components thus becomes a part of the key. If the secret components are cryptographically strong, the attacks become harder. In particular, it is harder to attack such primitives in the side-channel setting (though not impossible~\cite{SCARE}). Furthermore, structural attacks help to understand the security of structures themselves, independently of the specifics of the chosen components. In \ChapRef{feistel} I describe distinguishing and decomposition attacks against Feistel Networks. These results are based on the work done together with Léo Perrin~\cite{OurFeistel} and partly on the work done together with Léo Perrin and Alex Biryukov~\cite{OurKuz1}. My colleagues also studied the case of Substitution-Permutation-Networks~\cite{LeoSPN}.

The second reason comes from the white-box model. The seminal white-box implementations of AES and DES by Chow~\etal{}~\cite{ChowAES,ChowDES} are based on the composition of several small components into a single look-up table. Thus, the decomposition attacks pose a direct threat for the security of such implementations. Indeed, multiple decomposition attacks were given~\cite{AttackBillet,AttackMulder,AttackLepoint}. Another white-box construction called ASASA was proposed by Biryukov~\etal{}~\cite{ASASA}. It is a 2.5-round SPN with secret components. Most ASASA instances were broken in~\cite{ASASA1,ASASA2} resulting in a decomposition attack.

The third reason comes from the analysis of S-Boxes. S-Boxes are typically given as lookup tables. Usually, the designers describe the way they generated the S-Box. However, this is not always the case.
My colleagues Léo Perrin and Alex Biryukov wrote a seminal work on revealing the secret criteria behind S-Box designs. They called this research direction ``S-Box Reverse-Engineering''. In their work, they uncovered possible design criteria of the Skipjack S-Box. They continued developing the S-Box reverse-engineering techniques and found an interesting decomposition of the S-Box used in Russian standard cryptographic primitives.
I also contributed to the latter work~\cite{OurKuz1}. Later, we also found another interesting algebraic structure in that S-Box~\cite{OurKuz2}. I describe these decompositions in \ChapRef{kuz}. 
Furthermore, we found a surprising application of the developed S-Box decomposition techniques. We applied them to find a structure in an S-Box of \emph{mathematical} origin, the 6-bit APN permutation discovered by Dillon~\etal{}~\cite{DillonAPN}. I talk about this result in \ChapRef{apn}, which is based on our publication~\cite{OurAPN}.