White-box cryptography studies the security of cryptographic implementations in the white-box model. In this model, an adversary has full access to the implementation, in the form of a program or a circuit. She can, therefore, read or write memory at any time, perform precise fault attacks, analyze the program's control flow. Her goal depends on the security requirement. For white-box implementations of symmetric-key primitives, the most basic security requirement is the secrecy of the key. Such implementations are a long-standing open problem in cryptography. Starting from seminal works of Chow~\etal{}~\cite{ChowAES,ChowDES} in 2002, several constructions were proposed in the literature. Unfortunately, all were broken by practical attacks. However, in industry, such implementations are of large interest. Companies use white-box implementations with private designs. This led to a recent direction of applying side-channel attacks to white-box implementations. Bos~\etal{}~\cite{AttackBos} show that most implementations can be broken by a side-channel attack in a fully automated way.

In this part, I present the work I have done on white-box implementations of symmetric-key primitives. I explore further the space of automated attacks and provide provably secure protection against a new attack. This part is based on the joint work with Alex Biryukov~\cite{OurWhitebox}.