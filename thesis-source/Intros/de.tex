In this part, I present the work I have done on the design of symmetric-key primitives. The current trend in the design of cryptographic primitives is \emph{lightweight cryptography}. Lightweight cryptography targets small devices (e.g. microcontrollers, smart cards, RFID tags). These devices are very constrained in resources, and it is necessary to minimize memory usage, code size, energy consumption, time of computation. Lightweight cryptography lowers the security margin in order to obtain more efficient cryptosystems. Another reason supporting the lightweight trend is that many existing designs survived many years of cryptanalysis, and there were no breakthrough techniques in cryptanalysis for a long time. Therefore, designing a secure primitive is a problem with many existing solutions, and these solutions have to compete by other properties, e.g. \emph{lightweightness}.

\newcommand{\sparx}{\textsf{SPARX}}
\newcommand{\sparkle}{\textsf{SPARKLE}}
\newcommand{\esch}{\textsf{Esch}}
\newcommand{\schwaemm}{\textsf{Schwaemm}}
I participated in the design of the \sparx{} family of block ciphers~\cite{OurSPARX} and the \sparkle{} cryptographic permutation~\cite{OurSPARKLE}. I and my colleagues further used \sparkle{} and the sponge construction to design the hash function family \esch{} and authenticated encryption family \schwaemm{}. My main contributions were in the security evaluations of the designs. 

It has become a standard requirement for a symmetric-key design to include a proof against linear and differential cryptanalysis. The designers of AES, the current block cipher standard, used the so-called \emph{wide trail argument} for the proof. It is a quite effective argument for block ciphers with strong, small S-Boxes and strong, heavy linear layers. However, it fails for ARX-based designs, i.e. designs build only from addition, rotation, and \txor operations. Such designs have certain advantages, such as better resistance against side-channel attacks and better performance in software. My colleagues came up with a novel way to prove security against linear and differential attacks, called a \emph{long trail argument}. It is effective for designs using light linear layers and light but large S-Boxes. I designed an algorithm for applying the long-trail argument to a particular subset of SPN structures. We used this algorithm to evaluate a large class of linear layer candidates for the block cipher~\sparx{}, together with the division property~\cite{division} for security evaluation against integral cryptanalysis. I also used the algorithm to evaluate the security of \sparkle{}, a cryptographic permutation based on \sparx{}, that I and my colleagues designed for the NIST call for lightweight cryptography.